\section{גז אידאלי ואנרגיה קינטית}

\begin{cheatformula}[חוק הגז האידאלי]
\begin{center}
\begin{tabular}{lr}
    $\displaystyle P\,V = N\,k_{B}\,T$ 
    & 
    \scalebox{0.6}{$
    \begin{aligned}
        [P] &= 1\,\mathrm{atm} = 1.013 \times 10^{5}\,\mathrm{Pa} \\
        [V] &= 10^{3}\,\mathrm{L} = 1\,\mathrm{m^{3}} \\
        [T] &= \mathrm{K}
    \end{aligned}$}
\end{tabular}
\end{center}
אנרגיה קינטית ממוצעת של חלקיק גז אידאלי:
$\bigl\langle E \bigr\rangle \;=\;\frac{3}{2}\,k_{B}\,T$
\end{cheatformula}

\begin{cheatformula}[קשר יסודי של גז אידאלי]
    $$S(E,\,V,\,N) \;=\; Nk_B\ln \Big[ a \frac{V}{N^\tfrac{5}{2}} E^\tfrac{3}{2} \Big] \quad \text{\scriptsize $a = \frac{\left( \frac{4}{3} \pi m \right)^\tfrac{3}{2}}{h^3} e^\tfrac{5}{2}$}$$

\end{cheatformula}

\begin{cheatformula}[אוסילטור הרמוני] אנרגיה
$$E \;=\; \frac{p_{0}^{2}}{2\,m} \;+\; \frac{1}{2}\,m\,\omega^{2}\,x_{0}^{2}$$

האנרגיה הקינטית הממוצעת
\begin{align*}
\left\langle E_k \right\rangle = \int_{-\infty}^{\infty}\!\int_{-\infty}^{\infty}
\frac{dx_{0}\,dp_{0}}{2\pi\,\hbar}\;
\left(\frac{p_{0}^{2}}{2\,m}\right)\;P(x_{0},\,p_{0})
&= \\
\int_{-\infty}^{\infty}\!\int_{-\infty}^{\infty}
\frac{dx_{0}\,dp_{0}}{h}\;
\left(\frac{p_{0}^{2}}{2\,m}\right)\;P(x_{0},\,p_{0}) 
\end{align*}


\end{cheatformula}
