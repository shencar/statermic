\section{פרמיונים}
חלקיקים המחויבים לחוק פאולי. כלומר לא ניתן להבחין בין פרמיונים מאותו סוג. 
לא יתכן ששני פרמיונים מאותו סוג יהיו באותו מצב קוונטי\\הספין הכולל הוא בחצאי שלמים $\tfrac{1}{2},1,\tfrac{3}{2}\cdots$\\
דוגמאות - אלקטרונים, פרוטונים וניוטרונים.

\begin{cheatformula}[חוק האיסור של פאולי]
לא יתכן ששני פרמיונים מאותו סוג יהיו באותו מצב קוונטי
\end{cheatformula}

\begin{cheatformula}[פונקציית החלוקה הגרנד קנונית של פרמיונים]
    \begin{align*}
    \mathcal{Z}_\alpha \left(T,V,\mu\right) &= \sum_{N_\alpha = 0}^1 e^{-\beta \left( E_\alpha N_\alpha - \mu N_\alpha \right)} = 1 + e^{-\beta \left( E_\alpha - \mu \right)} \\
    \mathcal{Z} \left(T,V,\mu\right) &= \prod_\alpha \mathcal{Z}_\alpha \left(T,V,\mu\right) = \prod_\alpha \left( 1 + e^{-\beta \left( E_\alpha - \mu \right)} \right)
    \end{align*}
\end{cheatformula}

\begin{cheatformula}[התפלגות פרמי-דיראק]
מספר הפרמיונים הממוצע מאותו סוג תחת פוטנציאל במצב $\alpha$ 
    $$\left<N_\alpha \right>^{FD} = \frac{1}{e^{\beta \left(\epsilon_\alpha -\mu \right) }+1}$$
בטמפרטורה $T=0$ הפונקציה הופכת למדרגה עם קפיצה של:
$$\varepsilon_\alpha = \mu \left(T=0, V, N\right)$$
אם $\varepsilon_\alpha < \mu \left(T=0, V,N\right)$ אז בהכרח יש חלקיק במצב $\alpha$
\\
אם $\varepsilon_\alpha > \mu \left(T=0, V,N\right)$ אז בהכרח אין חלקיק במצב $\alpha$
\\
החלקיקים ממלאים את רמות האנרגיה עד \textbf{אנרגיית פרמי}
$$\varepsilon_F = \mu \left(T=0, V,N\right)$$
\end{cheatformula}

\begin{cheatformula}[מספר חלקיקים במערכת פרמיונים]
$$N = \sum_\alpha \left< N_\alpha \right>^{FD} = \sum_{n=0}^{\infty} \sum_{s=-\frac{1}{2}}^{\frac{1}{2}} \left< N_{n,s} \right>^{FD}$$
\end{cheatformula}

\begin{cheatformula}[מעבר מסכום לאינטגרל:]
אם לא ניתן לחשב את הסכום, ניתן לעבור לאינטגרל כאשר ההפרש בין איברים רצופים קטן מספיק. לדוגמה אם המקדם של $n$ קטן מספיק.
\begin{align*}
\sum_{n=1}^\infty \frac{2}{e^{\beta \hbar \omega \left(n+\frac{1}{2} \right) - \beta \mu}+1} \approx \int_{n=1}^\infty \frac{2 \, dn}{e^{\beta \hbar \omega \left(n+\frac{1}{2} \right) - \beta \mu}+1}
\end{align*}
התנאי: $\beta \hbar \omega \ll 1$
\end{cheatformula}

\begin{cheatformula}[צפיפות מצבים:]
כאשר מסכמים על מספר משתנים, נגדיר $\varepsilon(n_x,n_y,n_z)$ ונחשב את צפיפות המצבים $g(\varepsilon)$.
$$N = \sum_s \int_{\varepsilon_0}^\infty \frac{g(\varepsilon)}{e^{\beta(\varepsilon - \mu)} + 1} \, d\varepsilon$$
\end{cheatformula}

\begin{cheatformula}[חישוב אנרגיית פרמי $\varepsilon_F$:]\\
בגבול $T \to 0$, התפלגות פרמי-דיראק הופכת לפונקציית מדרגה:
$$\left< N_{T=0}(\varepsilon) \right>^{FD} = \begin{cases}
    1 & \varepsilon < \varepsilon_F \\
    0 & \varepsilon > \varepsilon_F
\end{cases}$$

לכן:
$$N = \sum_s \int_{\varepsilon_0}^{\varepsilon_F} g(\varepsilon) \, d\varepsilon$$
צריך לבדוק ש$\varepsilon_0 \to 0$ בגבול התרמודינמי ואז אפשר להציב 0 באינטגרל \\
מפה ניתן לחשב את אנרגיית פרמי $\varepsilon_F$
\end{cheatformula}

\begin{cheatformula}[טמפרטורת פרמי:]
$T_F = \frac{\varepsilon_F}{k_B}$
\end{cheatformula}

\begin{cheatformula}[אנרגיית המערכת בטמפרטורות נמוכות:]
\begin{align*} E \left(T,V,N\right) &= \sum_n \sum_s \left< N_{n,s} \right>^{FD} \cdot \varepsilon_{n,s} \\
    &\approx \sum_s \int_{\varepsilon_0}^{\infty} \varepsilon g\left(\varepsilon\right) \langle N(\varepsilon) \rangle^{FD} \, d\varepsilon
\end{align*}
את שאר החישוב מבצעים עם קירוב זומרפלד.
\end{cheatformula}

\begin{cheatformula}[גבולות]
הגבול הקלאסי של גז פרמיונים הוא כאשר:
$k_BT \gg \varepsilon_F$. \\
טמפרטורת פרמי: $T_F = \frac{\varepsilon_F}{k_B}$\\
כאשר $T \gg T_F$ המערכת מתנהגת קלאסית. (** חשוב לקחת בחשבון את התיקון של גיבס במקרה הקלאסי )
אבל כאשר $T \ll T_F$ אז המערכת מתנהגת קוונטית.
\end{cheatformula}
\begin{cheatformula}[קירוב זומרפלד]
\begin{align*}
&\int_{\varepsilon_0}^{\infty} H(\varepsilon)\,\bigl\langle N(\varepsilon)\bigr\rangle^{\mathrm{FD}}\,d\varepsilon
\approx \int_{\varepsilon_0}^{\varepsilon_F} H(\varepsilon)\,d\varepsilon \ + \\[6pt]
&+ \frac{\pi^2}{6}\,(k_B T)^2
\Biggl(
\left.\frac{dH}{d\varepsilon}\right|_{\varepsilon=\varepsilon_F}
- H(\varepsilon_F)\,
\left.\frac{d\ln g_{\mathrm{tot}}(\varepsilon)}{d\varepsilon}\right|_{\varepsilon=\varepsilon_F}
\Biggr)\\
&g_{\mathrm{tot}}\left(\varepsilon\right) = \sum_s g_s\left(\varepsilon\right) \\
\end{align*}

\end{cheatformula}