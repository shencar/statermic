\section{פרמיונים}


חלקיקים המחויבים לחוק פאולי. הספין הכולל הוא בחצאי שלמים $\tfrac{1}{2},1,\tfrac{3}{2}\cdots$. \\
דוגמאות: אלקטרונים, פרוטונים וניוטרונים.

\begin{cheatformula}[חוק האיסור של פאולי]
לא יתכן ששני פרמיונים מאותו סוג יהיו באותו מצב קוונטי
\end{cheatformula}



\begin{cheatformula}[פונקציית החלוקה הגרנד קנונית של פרמיונים]
    \begin{align*}
    \mathcal{Z}_\alpha \left(T,V,\mu\right) &= \sum_{N_\alpha = 0}^1 e^{-\beta \left( E_\alpha N_\alpha - \mu N_\alpha \right)} = 1 + e^{-\beta \left( E_\alpha - \mu \right)} \\
    \mathcal{Z} \left(T,V,\mu\right) &= \prod_\alpha \mathcal{Z}_\alpha \left(T,V,\mu\right) = \prod_\alpha \left( 1 + e^{-\beta \left( E_\alpha - \mu \right)} \right)
    \end{align*}
\end{cheatformula}

\begin{cheatformula}[התפלגות פרמי-דיראק]
מספר הפרמיונים הממוצע מאותו סוג במצב $\alpha$:
    $$\left<N_\alpha \right>^{FD} = \frac{1}{e^{\beta \left(\epsilon_\alpha -\mu \right) }+1}$$
\end{cheatformula}

\begin{cheatformula}[מספר חלקיקים במערכת]
$$N = \sum_\alpha \left< N_\alpha \right>^{FD} = \sum_{n=0}^{\infty} \sum_{s=-\frac{1}{2}}^{\frac{1}{2}} \left< N_{n,s} \right>^{FD}$$
\end{cheatformula}



\begin{cheatformula}[מעבר מסכום לאינטגרל]
כאשר ההפרש בין איברים רצופים קטן מספיק:
\begin{align*}
\sum_{n=1}^\infty \frac{2}{e^{\beta \hbar \omega \left(n+\frac{1}{2} \right) - \beta \mu}+1} \approx \int_{n=1}^\infty \frac{2 \, dn}{e^{\beta \hbar \omega \left(n+\frac{1}{2} \right) - \beta \mu}+1}
\end{align*}
התנאי: $\beta \hbar \omega \ll 1$
\end{cheatformula}

\begin{cheatformula}[צפיפות מצבים]
כאשר מסכמים על מספר משתנים, נגדיר $\varepsilon(n_x,n_y,n_z)$ ונחשב את צפיפות המצבים $g(\varepsilon)$:
$$N = \sum_s \int_{\varepsilon_0}^\infty \frac{g(\varepsilon)}{e^{\beta(\varepsilon - \mu)} + 1} \, d\varepsilon$$
\end{cheatformula}



\begin{cheatformula}[אנרגיית פרמי]
בטמפרטורה $T=0$, התפלגות פרמי-דיראק הופכת לפונקציית מדרגה:
$$\left< N_{T=0}(\varepsilon) \right>^{FD} = \begin{cases}
    1 & \varepsilon < \varepsilon_F \\
    0 & \varepsilon > \varepsilon_F
\end{cases}$$
החלקיקים ממלאים את רמות האנרגיה עד \textbf{אנרגיית פרמי}:
$$\varepsilon_F = \mu \left(T=0, V,N\right)$$
ולכן:
$$N = \sum_s \int_{\varepsilon_0}^{\varepsilon_F} g(\varepsilon) \, d\varepsilon$$
\end{cheatformula}

\begin{cheatformula}[טמפרטורת פרמי]
$$T_F = \frac{\varepsilon_F}{k_B}$$
\end{cheatformula}



\begin{cheatformula}[אנרגיית המערכת]
\begin{align*} 
E \left(T,V,N\right) &= \sum_n \sum_s \left< N_{n,s} \right>^{FD} \cdot \varepsilon_{n,s} \\
    &\approx \sum_s \int_{\varepsilon_0}^{\infty} \varepsilon g\left(\varepsilon\right) \langle N(\varepsilon) \rangle^{FD} \, d\varepsilon
\end{align*}
\end{cheatformula}

\begin{cheatformula}[קירוב זומרפלד]
עבור טמפרטורות נמוכות ($T \ll T_F$):
\begin{align*}
&\int_{\varepsilon_0}^{\infty} H(\varepsilon)\,\bigl\langle N(\varepsilon)\bigr\rangle^{\mathrm{FD}}\,d\varepsilon
\approx \int_{\varepsilon_0}^{\varepsilon_F} H(\varepsilon)\,d\varepsilon + \\[6pt]
&+ \frac{\pi^2}{6}\,(k_B T)^2
\Biggl(
\left.\frac{dH}{d\varepsilon}\right|_{\varepsilon=\varepsilon_F}
- H(\varepsilon_F)\,
\left.\frac{d\ln g_{\mathrm{tot}}(\varepsilon)}{d\varepsilon}\right|_{\varepsilon=\varepsilon_F}
\Biggr)
\end{align*}
כאשר $g_{\mathrm{tot}}\left(\varepsilon\right) = \sum_s g_s\left(\varepsilon\right)$
\end{cheatformula}



\begin{cheatformula}[גבולות קלאסי וקוונטי]
\textbf{גבול קלאסי:} $T \gg T_F$ (כלומר $k_BT \gg \varepsilon_F$) \\
במקרה זה המערכת מתנהגת קלאסית. \\
\textbf{חשוב:} יש לקחת בחשבון את התיקון של גיבס במקרה הקלאסי.

\textbf{גבול קוונטי:} $T \ll T_F$ \\
במקרה זה המערכת מתנהגת קוונטית ויש להשתמש בקירוב זומרפלד.
\end{cheatformula}