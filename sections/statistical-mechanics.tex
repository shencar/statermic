\section{מכניקה סטטיסטית - צברים ואנטרופיה}
\begin{cheatformula}[גדלים תרמודינמיים] \\
    גדלים אקסטנסיביים - גדלים שפרופורציונלי לגודל המערכת. לדוגמה:
    $E,V,N,S$ \\
    $S(\lambda E,\lambda V,\lambda N) = \lambda S(E,V,N)$ \\
    גדלים אינטנסיביים - גדלים שלא תלויים בגודל המערכת. לדוגמה:
    $T,P,\mu$ \\
    $T(\lambda S, \lambda V, \lambda N) = T(S, V, N)$
\end{cheatformula}

\begin{cheatformula}[חוקי התרמודינמיקה]
    \begin{enumerate}
        \item שתי מערכות בשיווי-משקל תרמודינמי עם מערכת שלישית, בשיווי-משקל זו עם זו
        \item חום היא צורה של אנרגיה, אנרגיה נשמרת
        \item האנרגיה של מערכת מורכבת מהחום שאגור בה והיכולת לבצע עבודה 
        $\dbar Q = du - \dbar W$
        \item במערכת סגורה האנטרופיה יכולה לגדול עם הזמן או לא להשתנות
    \end{enumerate}
\end{cheatformula}
\begin{cheatformula}[ספירת מצבים בדידים למערכת עם אנרגיה קבועה]\\
\begin{flushright}
\begin{enumerate}
    \item נכתוב את האנרגיה כתלות במצבים: 
    $$E(n_{1},\cdots,n_{N}) = E_{0}$$
    \item נחשב את הנפח שהמשטח חוסם, נחלק ביחידת נפח של מצב אחד — ונקבל את מספר המצבים בתוך המשטח
    $\Sigma(E, N)$
    \item נגזור לפי $E$ ונכפול ב-$\varepsilon$ (יחידת אנרגיה קטנה של הבעיה):
    $$\Omega(E,N) = \frac{d\Sigma(E, N)}{dE} \cdot \varepsilon$$    
    ונקבל את מספר המצבים המותרים בתחום 
    $\left[E - \frac{\varepsilon}{2},\, E + \frac{\varepsilon}{2} \right]$
    
\end{enumerate}
\end{flushright}

\begin{cheatformula}[אנטרופיה]
    $$S(E,\,V,\,N) \;=\; k_{B}\,\ln \Omega(E,V,N)$$
\end{cheatformula}
\end{cheatformula}
\begin{cheatformula}[טמפרטורה]
     \begin{align*}
        \frac{1}{T} \;=\; \Bigl(\tfrac{\partial S}{\partial E}\Bigr)_{V,N} &\qquad
        \beta = \frac{1}{\tau} \;=\; \frac{1}{k_{B}\,T} \;=\; \frac{\partial\sigma}{\partial u}\, \\\\
        T \to 0 &\iff S \to 0
    \end{align*}
\end{cheatformula}

\begin{cheatformula}[צברים]
\begin{itemize}
 \item מיקרוקנוני - $E,V,N$ (מערכת מבודדת עם אנרגיה, נפח, ומספר חלקיקים קבועים)
 $$
 P \left(\mu\right) = \begin{cases}
     \frac{1}{\Omega \left( E,V,N \right)} & E\left(\mu\right)=E, V\left(\mu\right)=V,N\left(\mu\right)=N\\
     0 & o.w.
 \end{cases}
 $$
 \item קנוני - $T,V,N$ (מערכת במגע עם אמבט חום בטמפרטורה קבועה, נפח, ומספר חלקיקים קבועים)
 $$
 P \left(\mu\right) = \begin{cases}
     \frac{1}{Z \left( T,V,N \right)} e^{-\beta E\left(\mu\right)} &  V\left(\mu\right)=V,N\left(\mu\right)=N\\
     0 & o.w.
 \end{cases}
 $$
 \item גרנד קנוני - $T,V,\mu$ (מערכת במגע עם אמבט חום ומאגר חלקיקים, עם טמפרטורה, נפח, ופוטנציאל כימי קבועים)
 $$
 P \left(\overline{\mu}\right) = \begin{cases}
     \frac{1}{\mathcal{Z} \left( T,V,\mu \right)} e^{-\beta\left( E\left(\overline{\mu}\right) - \mu N\left(\overline{\mu}\right)  \right)} &  V\left(\overline{\mu}\right)=V\\
     0 & o.w.
 \end{cases}
 $$
\end{itemize}
\end{cheatformula}

\begin{cheatformula}[חישוב גדלים תרמודינמיים בצבר המיקרוקנוני]
\begin{enumerate}
        \item נכתוב את האנרגיה כתלות במיקרו-מצבי המערכת
        \item נחשב את $\Omega \left( E,V,N \right)$ ע"י ספירת המצבים כתלות בגדלים
        \item נחשב את הקשר היסודי $S\left( E,V,N \right) = k_B \ln \Omega \left( E,V,N \right)$
        \item נגזור כל גודל תרמודינמי 
\end{enumerate}
\end{cheatformula}

\begin{cheatformula}[הגבול התרמודינמי]
    \begin{align*}
E &\to \infty, \quad V \to \infty, \quad N \to \infty \\
\frac{E}{V} &\to \text{const}, \quad \frac{N}{V} \to \text{const}, \quad \frac{E}{N} \to \text{const}
\end{align*}
\end{cheatformula}

\begin{cheatformula}[דיפרנציאלים]\\
דיפנרציאל שלם אם ורק אם
$ \left( \frac{\partial f_x}{\partial y} \right) = \left( \frac{\partial f_y}{\partial x} \right)$
\[
dE = \underbrace{TdS}_{\dbar Q} + \underbrace{(-PdV + \mu dN)}_{\dbar W}
\]
\end{cheatformula}
