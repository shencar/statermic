\section{מכניקה סטטיסטית - צברים ואנטרופיה}
\begin{cheatformula}[גדלים תרמודינמיים] \\
    גדלים אקסטנסיביים - גדלים שפרופורציונלי לגודל המערכת. לדוגמה:
    $E,V,N,S$ \\
    $S(\lambda E,\lambda V,\lambda N) = \lambda S(E,V,N)$ \\
    גדלים אינטנסיביים - גדלים שלא תלויים בגודל המערכת. לדוגמה:
    $T,P,\mu$ \\
    $T(\lambda S, \lambda V, \lambda N) = T(S, V, N)$
\end{cheatformula}

\begin{cheatformula}[ספירת מצבים בדידים למערכת עם אנרגיה קבועה]\\
\begin{flushright}
\begin{enumerate}
    \item נכתוב את האנרגיה כתלות במצבים: 
    $$E(n_{1},\cdots,n_{N}) = E_{0}$$
    \item נחשב את הנפח שהמשטח חוסם, נחלק ביחידת נפח של מצב אחד — ונקבל את מספר המצבים בתוך המשטח
    $\Sigma(E, N)$
    \item נגזור לפי $E$ ונכפול ב-$\varepsilon$ (יחידת אנרגיה קטנה של הבעיה):
    $$\Omega(E,N) = \underset{g}{\underbrace{\frac{d\Sigma(E, N)}{dE}}}  \cdot \varepsilon$$    
    ונקבל את מספר המצבים המותרים בתחום 
    $\left[E - \frac{\varepsilon}{2},\, E + \frac{\varepsilon}{2} \right]$
    
\end{enumerate}
\end{flushright}

\begin{cheatformula}[אנטרופיה]
    $S(E,\,V,\,N) \;=\; k_{B}\,\ln \Omega(E,V,N)$
\end{cheatformula}
\end{cheatformula}
\begin{cheatformula}[טמפרטורה]
     $
        \frac{1}{T} \;=\; \Bigl(\tfrac{\partial S}{\partial E}\Bigr)_{V,N} \qquad
        \beta = \frac{1}{k_{B}\,T}
    $
\end{cheatformula}

\begin{cheatformula}[הגבול התרמודינמי]
    \begin{align*}
E &\to \infty, \quad V \to \infty, \quad N \to \infty \\
\frac{E}{V} &\to \text{const}, \quad \frac{N}{V} \to \text{const}, \quad \frac{E}{N} \to \text{const}
\end{align*}
\end{cheatformula}

\begin{cheatformula}[דיפרנציאלים]\\
דיפנרציאל שלם אם ורק אם
$ \left( \frac{\partial f_x}{\partial y} \right) = \left( \frac{\partial f_y}{\partial x} \right)$
\[
dE = \underbrace{TdS}_{\dbar Q} + \underbrace{(-PdV + \mu dN)}_{\dbar W}
\]
* בתהליך התפשטות איטי, האנטרופיה לא משתנה, אז $dS = 0$.
\end{cheatformula}
