\section{גדלים תרמודינמניים והתמרת לז'נדר}

\begin{cheatformula}[לחץ]
$$P \;=\; T \,\Bigl(\frac{\partial S}{\partial V}\Bigr)_{E,N} \;=\; -\,\Bigl(\frac{\partial E}{\partial V}\Bigr)_{S,N}$$
\end{cheatformula}

\begin{cheatformula}[פוטנציאל כימי]
$$\mu \;=\; -\,T \,\Bigl(\frac{\partial S}{\partial N}\Bigr)_{E,V} \;=\; \Bigl(\frac{\partial E}{\partial N}\Bigr)_{S,V}$$
\end{cheatformula}

\begin{cheatformula}[התמרת לז'נדר]\\
 במשתנה יחיד$Y(X) \to \psi(P)$
\begin{enumerate}
    \item נוודא ש- $Y(X)$ קעורה ממש או קמורה ממש בקטע
    \item נגזור $P(X) = \frac{dY}{dX}$
    \item נהפוך את הקשר ונקבל $X(P)$
    \item נציב את הקשר ההפוך בביטוי $\psi(P) = Y(X(P)) - PX(P)$
\end{enumerate}

ב- $n$ משתנים $Y(\vec{X}) \to \psi(\vec{P})$
\begin{enumerate}
    \item נוודא שהפונקציה $Y(\vec{X})$ קמורה ממש או קעורה ממש בתחום בו מתבצעת הטרנספורמציה.
    \item נגזור: עבור כל משתנה $X_i$, נגדיר את $P_i = \frac{\partial Y}{\partial X_i}$.
    \item נהפוך את מערכת הקשרים ונקבל את $\vec{X}(\vec{P})$ — כלומר, מבטאים את כל $X_i$ כפונקציה של המשתנים $P_i$.
    \item נגדיר את הפונקציה החדשה: \[
        \psi(\vec{P}) = \left[\sum_{i=1}^n P_i X_i(\vec{P})\right] - Y\big(\vec{X}(\vec{P})\big)
    \]
\end{enumerate}

\end{cheatformula}

\begin{cheatformula}[יחסי מקסוול]\\
\begin{align*}
    \Bigl(\tfrac{\partial T}{\partial V}\Bigr)_{S,N} &= - \Bigl(\tfrac{\partial P}{\partial S}\Bigr)_{V,N} &= \Bigl(\tfrac{\partial^2 E}{\partial S \partial V}\Bigr) \\ 
    * \Bigl(\tfrac{\partial P}{\partial T}\Bigr)_{V,N} &= \Bigl(\tfrac{\partial S}{\partial V}\Bigr)_{T,N} &= - \Bigl(\tfrac{\partial^2 F}{\partial T \partial V}\Bigr) \\ 
    \Bigl(\tfrac{\partial T}{\partial P}\Bigr)_{S,N} &= \Bigl(\tfrac{\partial V}{\partial S}\Bigr)_{P,N} &=  \Bigl(\tfrac{\partial^2 H}{\partial S \partial P}\Bigr)\\
    \Bigl(\tfrac{\partial S}{\partial P}\Bigr)_{T,N} &= - \Bigl(\tfrac{\partial V}{\partial T}\Bigr)_{P,N} &= \Bigl(\tfrac{\partial^2 G}{\partial T \partial P}\Bigr)\\
    \Bigl(\tfrac{\partial \mu}{\partial P}\Bigr)_{S,N} &=  \Bigl(\tfrac{\partial V}{\partial N}\Bigr)_{S,P} &= \Bigl(\tfrac{\partial^2 H}{\partial P \partial N}\Bigr)
\end{align*}  
* ראינו בתירגול (מה שלא ראינו בתירגול מומלץ להראות את החישוב) 
\end{cheatformula}

\begin{cheatformula}[חישוב יחסים בין דיפרנציאלים]
    דוגמה עם אנתלפיה:
\begin{align*} 
    dH = \left(\frac{\partial H}{\partial T}\right)_{P,N} dT + \left(\frac{\partial H}{\partial P}\right)_{T,N} dP + \left(\frac{\partial H}{\partial N}\right)_{T,P} dN
\end{align*}
אם $H$ ו-$N$ קבועים, אז
\begin{align*}
    0 &= \left(\frac{\partial H}{\partial T}\right)_{P,N} dT + \left(\frac{\partial H}{\partial P}\right)_{T,N} dP \\
     & - \left(\frac{\partial T}{\partial P}\right) = \frac{\left(\frac{\partial H}{\partial P}\right)_{T,N}}{\left(\frac{\partial H}{\partial N}\right)_{T,P}} 
\end{align*}
\end{cheatformula}
