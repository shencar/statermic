\section{הצבר הקנוני - $T,V,N$}
מערכת סגורה שמבודדת מהסביבה, אבל יכולה להחליף אנרגיה עם הסביבה.
למשל, מערכת שצמודה לאמבט חום.
\begin{cheatformula}[ההסתברות למצוא מיקרו-מצב $\mu$]
$$
 P \left(\mu\right) = \begin{cases}
     \frac{1}{Z \left( T,V,N \right)} e^{-\beta E\left(\mu\right)} &  V\left(\mu\right)=V,N\left(\mu\right)=N\\
     0 & o.w.
 \end{cases}
 $$\end{cheatformula}
\begin{cheatformula}[פונקציית החלוקה]
$$Z\left(T,V,N\right) = \sum_\text{מצבים} e^{-\beta E_\text{מצב}}$$
\end{cheatformula}
\begin{cheatformula}[האנרגיה החופשית של הלמהולץ]
$$F=-k_BT\ln{Z}$$    
\end{cheatformula}

\begin{cheatformula}[האנרגיה הממוצעת של מערכת]
\begin{align*}
    \left< E(T,V,N) \right> = - \Bigl(\tfrac{\partial \ln{Z}}{\partial \beta}\Bigr)_{V,N} = 
    k_B T^2 \Bigl(\tfrac{\partial \ln{Z}}{\partial T}\Bigr)_{V,N}
\end{align*}

\begin{cheatformula}[הפלקטואציה של אנרגיה]
\begin{align*}
    \left( \Delta E \left(T,V,N\right) \right)^2 = - \left( \frac{\partial \left< E \right>}{\partial \beta}\right)_{V,N} =
    k_B T^2 \Bigl(\tfrac{\partial \left< E \right>}{\partial T}\Bigr)_{V,N}
\end{align*}
\end{cheatformula}

\begin{cheatformula}[משפט החלוקה השווה]
    כל איבר ריבועי בקואורדינטות בביטוי לאנרגיה של אופני תנודה תורם $\frac{1}{2}k_BT$ לאנרגיה של המערכת
\end{cheatformula}

\end{cheatformula}

\begin{cheatformula}[חישוב גדלים תרמודינמיים בצבר הקנוני]
    \begin{enumerate}
        \item נכתוב את האנרגיה של המערכת כתלות במצבים
        \item נחשב את $Z\left(T,V,N\right) = \sum_{\hat{\mu}} e^{-\beta E\left(\hat{\mu}\right)}$
        לשים לב אם אנחנו רוצים להבחין בין חלקיקים. אם לא אז נרצה להוסיף את התיקון של גיבס.
        \item נחשב את האנרגיה החופשית של הלמהולץ
        \item נגזור גדלים תרמודינמיים. \\
        לדוגמה כדי לחשב אנטרופיה נחשב את הדיפרנציאל:
        $$ dF = dE - SdT - TdS$$
        נציב את דיפרנציאל האנרגיה:
        $$dF = \left( TdS -PdV + \mu dN \right) - SdT - TdS$$
        נשים לב אילו גדלים לא משתנים ונאפס את הדיפרנציאל שלהם:
        $dF = -SdT + \mu \underset{0}{\underbrace{dN}}$
        ונקבל את הקשר:
        $\left< S \left(T,N\right) \right> = - \left( \frac{\partial F}{\partial T} \right)_N$
    \end{enumerate}
\end{cheatformula}
