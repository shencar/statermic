\section{קרינת גוף שחור}
\begin{cheatformula}[ביטוי כללי לאנרגיה אלקטרומגנטית]
\begin{align*}
    E = \frac{1}{2} \int_V \left( \epsilon_0 \abs{\vec{E}(x, y, z, t)}^2 + \frac{1}{\mu_0} \abs{\vec{B}(x, y, z, t)}^2 \right) \, dV
\end{align*}
\end{cheatformula}
\begin{cheatformula}[משוואות מקסוול]
    \begin{align*}
    & \div \vec{E} = \frac{\rho}{\epsilon_0} 
    && \div \vec{B} = 0 \\
    &\ \curl \vec{E} = -\frac{\partial \vec{B}}{\partial t}
    && \curl \vec{B} = \mu_0 \vec{J} + \mu_0 \epsilon_0 \frac{\partial \vec{E}}{\partial t}
\end{align*}

\begin{cheatformula}[משוואות הגלים]
\begin{align*}
    \nabla^2 \vec{E} + \frac{1}{c^2} \frac{\partial^2 \vec{E}}{\partial t^2} = 0 \qquad \vec{\nabla} \cdot \vec{E} = 0 \\
    \nabla^2 \vec{B} + \frac{1}{c^2} \frac{\partial^2 \vec{B}}{\partial t^2} = 0 \qquad \vec{\nabla} \cdot \vec{B} = 0
\end{align*}
פתרון עם תנאי שפה מתאפסים (יש לנו חופש לבחור תנאי שפה):
\begin{align*}
    \vec{E} &= \sum_{\scriptsize n_x,n_y,n_z} \vec{E}_{\vec{n}}(t) \sin\left(\frac{n_x \pi x}{L_x}\right) \sin\left(\frac{n_y \pi y}{L_y}\right) \sin\left(\frac{n_z \pi z}{L_z}\right) \\
    \vec{B} &= \sum_{\scriptsize n_x,n_y,n_z} \vec{B}_{\vec{n}}(t) \sin\left(\frac{n_x \pi x}{L_x}\right) \sin\left(\frac{n_y \pi y}{L_y}\right) \sin\left(\frac{n_z \pi z}{L_z}\right)
\end{align*}
\end{cheatformula}

אופן תנודה של גל בתדירות $\omega$:
\[
\omega = c \abs{\vec{k}} = \pi c \sqrt{ \left( \frac{n_x}{L_x} \right)^2 + \left( \frac{n_y}{L_y} \right)^2 + \left( \frac{n_z}{L_z} \right)^2 }
\]

\end{cheatformula}

\begin{cheatformula}[צפיפות אנרגיה ספקטרלית]
$
    u(T,\omega) = \frac{1}{V} \left( \frac{\partial E}{\partial \omega} \right)_{T,V}
$
\end{cheatformula}

\begin{cheatformula}[חוק ריילי-ג'ינס]
$
    u(T,\omega) = \frac{k_B T}{\pi^2 c^3} \omega^2
$ \\
(מתאים רק לתדירויות מתחת לאולטרה סגול)

\end{cheatformula}

\begin{cheatformula}[חוק פלאנק]
$
    u(T,\omega) = \frac{\hbar}{\pi^2 c^3} \frac{\omega ^3}{e^{\frac{\hbar \omega}{k_B T}}-1}
$ \\
כל אופן תנודה יכול לקבל אנרגיות המקיימות את המשוואה:
\begin{align*}
E_{n_x,n_y,n_z,p} &= s_{n_x,n_y,n_z,p} \cdot \hbar \omega(n_x,n_y,n_z) \\
s_{n_x,n_y,n_z,p} &\in \mathbb{N} \qquad p \in \{1,2\}
\end{align*}

\end{cheatformula}
\begin{cheatformula}[חוק סטפן-בולצמן]
סך כל כמות האנרגיה שפולט גוף שחור ע"י קרינה ליחידת זמן וליחידת שטח
\begin{align*}
    I = \frac{1}{A} \frac{E_{body}}{dt} = -\sigma T^4 \qquad \sigma = \frac{\pi ^2 k_B ^2}{60\hbar ^3 c^2} \approx 5.67 \cdot 10^{-8}
\end{align*}
\end{cheatformula}

\begin{cheatformula}[חוק ההסחה של וין]
    אורך הגל $\lambda_{\max}$ שבו גוף שחור יפלוט את מירב הקרינה
    $$\lambda_{\max} T = \alpha \qquad \alpha = 2.90 \cdot 10^-3 \, m$$
    ניתן למצוא את $\alpha$ ע"י מציאת המקסימום של חוק פלאנק 
\end{cheatformula}

\begin{cheatformula}[פונקציית החלוקה של גוף שחור]
\begin{align*}
Z_{n_x,n_y,n_z,p} \left(T,V\right) &= \sum_{s=0}^\infty e^{-\beta \epsilon_{n_x,n_y,n_z,p}(s)} \\
&= \frac{1}{1-e^{-\frac{\beta \pi \hbar c}{L}\sqrt{n_x^2 + n_y^2 + n_z^2}}} \\
Z\left(T,V\right) &= \prod_{\tiny n_x,n_y,n_z,p} Z_{\tiny n_x,n_y,n_z,p} \left(T,V\right) \\
&= \prod_{\tiny n_x,n_y,n_z} \left(\frac{1}{1-e^{-\frac{\beta \pi \hbar c}{L}\sqrt{n_x^2 + n_y^2 + n_z^2}}}\right)^2
\end{align*}
\end{cheatformula}
