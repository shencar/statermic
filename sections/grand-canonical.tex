\section{הצבר הגרנד-קנוני - $T,V,\mu$}
הצבר שמתאר מערכות \textbf{פתוחות} בשיווי-משקל תרמודינמי.
נפח תחום, אבל אנרגיה ומספר חלקיקים שלא נשמרים.
רק V גודל שמור, $T,\mu$ קבועים
בגבול התרמודינמי יש בדיוק $\langle N \rangle$ חלקיקים.

\begin{cheatformula}[ההסתברות למצוא מיקרו-מצב $\mu$]
$$
 P \left(\overline{\mu}\right) = \begin{cases}
     \frac{1}{\mathcal{Z} \left( T,V,\mu \right)} e^{-\beta\left( E\left(\overline{\mu}\right) - \mu N\left(\overline{\mu}\right)  \right)} &  V\left(\overline{\mu}\right)=V\\
     0 & o.w.
 \end{cases}
 $$
 ** שימו לב שאם יש כמה מיקרומצבים שנותנים את אותו גודל צריך לסכום אותם, לדוגמה:
 $$ P\left(N = N_0\right) = \sum_{\bar{\mu}_{N_0}} \frac{1}{\mathcal{Z}\left(T,\mu\right)} e^{-\beta\left( E\left(\bar{\mu}_{N_0}\right) - \mu N_0  \right)}  $$
\end{cheatformula}

\begin{cheatformula}[פונקציית החלוקה הגרנד-קנונית]
\[
\mathcal{Z} \left( T, V,\mu \right) = \sum_{\bar{\mu}} e^{-\beta \left( E(\bar{\mu}) - \mu N(\bar{\mu}) \right) } 
\]
לפעמים אפשר לחשב מתוך פונקציית החלוקה של הצבר הקנוני:
\[
\mathcal{Z} \left( T, V,\mu \right) = \sum_{N} Z \left( T, V, N \right) e^{\beta \mu N} \\
\]
\[
P \left(N=N_0\right) = \frac{Z \left( T, V, N_0 \right) }{\mathcal{Z} \left( T,V,\mu \right)} e^{-\beta \mu N_0}
\]
\end{cheatformula}


\begin{cheatformula}[הפוטנציאל הגרנד-קנוני]
\[
\Omega \left( T,V,\mu \right) = -k_B T \ln \mathcal{Z} \left( T,V,\mu \right)
\]
\end{cheatformula}

קשרים חשובים:
    \begin{align*}
\langle E \left( T,V,\mu \right) \rangle &= - \left( \frac{\partial \ln \mathcal{Z}}{\partial \beta} \right)_{V,e^{\beta \mu }} &&\: \left( \Delta E \right) ^2 = - \left( \frac{\partial \left< E \right> }{\partial \beta}  \right)_{V,e^{\beta \mu }} \\[1em]
\langle N \left( T,V,\mu \right) \rangle &= \frac{1}{\beta} \left( \frac{\partial \ln \mathcal{Z}}{\partial \mu} \right)_{V,T } &&\: \left( \Delta N \right) ^2 = \frac{1}{\beta} \left( \frac{\partial \left< N \right> }{\partial \mu}  \right)_{V,T}
\end{align*}

\begin{cheatformula}[חישוב גדלים תרמודינמיים בצבר הגרנד-קנוני]
הצבר הגרנד-קנוני רלוונטי בעיקר בבעיות שיש אתרים (מצבים, מדפים, רמות אנרגיה)

\begin{enumerate}
    \item נחשב את האנרגיה ומספר החלקיקים כתלות במצבי המערכת
    \item נחשב את פונקציית החלוקה הגרנד-קנונית $\mathcal{Z}\left( T,V,\mu \right)$
    \item נחשב את הפוטנציאל הגרנד-קנוני $\Omega\left( T,V,\mu \right)$
    \item (לרוב) נחשב את $\left< N \left( T,V,\mu \right) \right>$ וננסה לחלץ את $\mu \left( T,V, \left <N \right> \right)$
    \item נגזור מפונקציית החלוקה כל גודל תרמודינמי
\end{enumerate}

\end{cheatformula}

\begin{cheatformula}[התיקון של גיבס]
    בטמפרטורה גבוהה לחלקיקים יש מספר מאוד גדול של מיקרו-מצבים. אז אנחנו רוצים להזניח את הסיכוי ששני חלקיקים יהיו באותו מיקרו מצב
    $$Z\left(T,N\right) \approx \frac{1}{N!} \prod_{i=1}^{N}Z_i\left(T \right)= \frac{1}{N!} Z_1\left(\left(T \right)\right)^N$$
לא נשתמש בקירוב במערכות עם טמפרטורות נמוכות או צפיפויות גבוהות.
\end{cheatformula}

** עצה: כשמחשבים גדלים כדאי לבדוק גבולות, לדוגמה:
$\lim_{T\to 0} S = 0$
