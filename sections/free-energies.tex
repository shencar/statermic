\section{אנרגיות חופשית / פוטנציאלים תרמודינמיים}
* כל האנרגיות החופשיות הן פונקציות אקסטנסיביות ואדיטיביות.

\begin{cheatformula}[שיווי משקל תרמודינמי]\\
במערכת מבודדת ש-$\alpha$ גודל משתנה בה
$$\left( \frac{\partial S}{\partial \alpha} \right)_{E,V,N,\cdots} = 0$$
\end{cheatformula}

\begin{cheatformula}[אנרגיה פנימית]
הקשר היסודי
$$E = E(S,\,V,\,N)$$
\end{cheatformula}

\begin{cheatformula}[האנרגיה החופשית של הלמהולץ]
$T$ קבועה
בשיווי משקל תרמודינמי $F$ מינימלי
$$ F(T,\,V,\,N) \;=\; E \;-\; T\,S$$
\end{cheatformula}

\begin{cheatformula}[אנתלפיה]
$P$ קבוע
$$H(S,\,P,\,N) \;=\; E \;+\; P\,V$$
\end{cheatformula}

\begin{cheatformula}[האנרגיה החופשית של גיבס]
$T,P$ קבועים
$$G(T,\,P,\,N) \;=\; E \;-\; T\,S \;+\; P\,V$$
\end{cheatformula}


\begin{cheatformula}[הפוטנציאל הגרנד-קנוני]
$T,\mu$ קבועים
$$\Omega(T,\,V,\,\mu) \;=\; E \;-\; T\,S \;-\; \mu\,N$$
(נכון גם לערכים ממוצעים)
\end{cheatformula}
